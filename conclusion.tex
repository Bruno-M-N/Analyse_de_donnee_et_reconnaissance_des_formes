Nous avons, à travers cette étude, pu caractériser différentes méthodes de classifications et de factorisation d'un jeu de données. Si ces méthodes permettent toutes d'arriver à un même but, leur efficacité dépend de différents paramètres que nous avons mis en exergue, et elles ont toutes leurs limites. Ainsi, pour la méthode des centres mobiles,  nous avons vu que son utilisation de l'aléatoire pour placer les centres peut mener à une classification non conforme
aux consignes. Nous avons aussi vu que la méthodes CAH est très dépendante du calcul de distances utilisées.

De même, certaines méthodes de factorisation sont exclusive à des données quantitatives ou qualitatives, ce qui peut nécessiter un traitement de données au préalable. Toutes ces informations nous permettent de juger qu'aucune méthode n'est objectivement meilleure, mais qu'un choix judicieux des méthodes à appliquer est nécessaire au cas par cas pour obtenir un traitement de données adapté.