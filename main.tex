\documentclass{rapportECL}

\usepackage{minted}
%https://www.overleaf.com/learn/latex/Code_Highlighting_with_minted

%https://www.overleaf.com/latex/examples/source-code-highlighting-with-minted-in-latex/qphhfvnsddbs


 
%\usepackage{listingsutf8} %Display code in LaTeX, using the lstlisting environment
% https://tex.stackexchange.com/questions/312429/an-error-appears-saying-that-utf8-and-listing-are-not-compatible?rq=1


\title{Méthode Mixte : classification et analyse factorielle} %Titre du fichier

\begin{document}

%----------- Informations du rapport ---------

\titre{Méthode Mixte : classification et analyse factorielle} %Titre du fichier .pdf
\UE{UE INF a 4-EG} %Nom de la UE
\sujet{Analyse de Données} %Nom du sujet

\enseignant{Emmanuel\textsc{Dellandréa}} %Nom de l'enseignant

\eleves{Bruno \textsc{Moreira Nabinger} \\
		Clément \textsc{Vinot} } %Nom des élèves

%----------- Initialisation -------------------
        
\fairemarges %Afficher les marges
\fairepagedegarde %Créer la page de garde
\tabledematieres %Créer la table de matières

%------------ Corps du rapport ----------------


\section{Introduction} 

L'objectif général de ce 

\section{Cahier des charges}

\section{Principe des Solutions}

Des modules différents ont été développés pour chacune des 

\section{Implementation}

L'implementation du programme a été faite avec la langage Python 3.7. Le documentation a été fait en utilisant docstrings et des commentaires. On a aussi fait le design du code de tel façon  à aider la compréhension, en choisissant les nome des variables, attributs et méthodes qui explicitent la solution. Quelques méthodes ont été nommés d'après les fichiers disponibles dans la plateforme Pédagogique.

L'archive BE3-remarques-fusion.pdf disponibles dans la plateforme Pédagogique a été utilise pour réaliser quelques optimisations dans le code.

Le code est montré dans le topique suivant:

\subsection{Code}

\inputminted[linenos=True]{python}{scr/codage.py}

\inputminted[linenos=True]{python}{scr/acp.py}

\inputminted[linenos=True]{python}{scr/cah.py}

\subsection{Analyse des résultats}

\section{Conclusion}

\end{document}